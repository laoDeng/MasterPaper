%%This is a very basic article template.
%%There is just one section and two subsections.
\documentclass[UTF8]{ctexart}
\title{第六章 JESD204B确定性时延}
\author{陈登}
\date{\today}

\bibliographystyle{plain}
\usepackage{graphicx}
\usepackage{float}
\usepackage{amsmath}
\usepackage{geometry}
\usepackage{fontspec}
\usepackage{algorithm}
\usepackage{algorithmicx}
\usepackage{algpseudocode}

\geometry{a4paper,centering,scale=0.9}
\usepackage[format=hang,font=small,textfont=it]{caption}
\usepackage[toc,page,title,titletoc,header]{appendix}
\usepackage[nottoc]{tocbibind}

\begin{document}

\section{JESD204B确定性时延}

JESD204B规定的两种需要实现确定性时延的Subclass,主要需要设计两部分的对齐策略。
一是frame层面的确定性时延,这一部分主要是保证在本地multiframe时钟控制下,接收到的数据能同步的传输到接下来的模块中。
二是本地multiframe时钟层面的确定性时延,这一部分主要保证各个设备间的本地multiframe时钟能够同步,从而使整个系统之间能够有同样的收发节拍,能够将各个模块、设备之间的不确定性降到最低。

\subsection{frame层面的确定性时延实现}

link之间的确定性时延定义在收端和发端的以frame为基础的并行数据层面,即保证在frame时钟域。
不同link间的时延保证需要以frame时钟周期为单位,并且能够通过寄存器配置完成具体时延的设置。
这样的设置需要在上电或者link重连阶段能够有一定的重复性。

为了达成确定性时延,需要满足以下两个条件:

\begin{itemize}
\item 对于发端设备,所有lane之间的ILA序列产生必须同时进行,这也同时保证了数据流同时产生。
\item 对于收端设备,每个lane收到的数据必须在各个物理通道间设置缓存。收端缓冲需要在一个预先定义好的时刻,同时释放所有lane的数据。这个特殊时刻指的是可编程数量frame周期RBD\footnote{Rx Buffer Delay}(在本地multiframe时钟边界后)。
\end{itemize}

ILAS生成和接收端缓冲释放对齐,都是以接收端和发送端的本地multiframe时钟为标准。
所以,确定性时延的最小不确定性依赖于接收端和发送端之间本地multiframe时钟的对齐程度。

为了实现确定性时延的协议要求,系统设计时需要保证以下几点:

\begin{itemize}
\item multiframe长度必须大于link间可能存在的最大时延。
\item $RDB * T_f$的值必须大于link间可能存在的最大时延。
\item $RDB$的值,依照frame周期,必须介于1到K之间。
\end{itemize}

以上三条的目的是保证RBD够大,能够使接收端缓冲在释放前,所有lane的发送端数据被收到。
最终JESD204B link间的时延为$RDB * T_f$。
JESD204B link间的确定性时延要求接收端设备能够缓冲收到所有lane的ILA序列或者用户数据,在接收端缓冲能被释放前。
缓冲必须在本地multiframe时钟边界过后,RBD个frame周期之后释放。
为了能够释放缓冲,需要满足所有lane都有“有效数据”存在各自的接收端缓冲中,这里有“效数据”指:

\begin{itemize}
\item 如果ILA序列发送到了接收端缓冲中,那么有效数据指的是ILA序列的开端。
\item 如果ILA序列没有发送到接收端缓冲中,那么有效数据指的就是完整ILA序列之后出现的样本数据。在这种情况下接收端缓冲将会比第一种情况迟4个multiframe释放。
\end{itemize}

link间的时延可由下式表示,这也是所谓的确定性时延需要保证的确定部分:

$$Delay_{LINK} = \delta T_{LMFC} = TX \quad delay + Lane \quad Delay + RX \quad delay$$

其中,$TX \quad delay$指的是从发送端ILA序列产生,并出现在发送端SerDes的输出端口上的时间;$Lane \quad delay$指的是穿过外部物理信道的时延;$RX \quad delay$指的是从接收端SerDes输入到通入缓冲输出的时延。ILA序列的开端或者用户数的开端,将会在本地multiframe时钟边加上RBD个帧时钟周期时出现在缓冲输出;$\delta T_{LMFC}$指的是link间的全部时延,指的是发送端本地multiframe时钟上升沿到接收端本地multiframe时钟加上$T_f * RBD$上升沿之间的时延。

接收端缓冲的最小大小取决于最早可能到达接收端缓冲输入的数据和下一个接收端缓冲释放条件产生之间的时差。

Subclass 0类设备并不支持确定性时延。

\section{本地multiframe时钟层面的确定性时延}

由frame层面的确定性时延处理方法可以总结出来,各个设备之间的同步最关键在于同步各自的本地multiframe时钟边界,即需要各个设备能够将自己的本地multiframe时钟对准到所需要的边界上。
Subclass 1类和Subclass 2类设备最大的区别就在于本地multiframe时钟的对齐方式上,Subclass 1类设备需要额外的外部时钟信号SYSREF来保证同步,而Subclass 2类设备则需要通过自身的SYNC~信号来保证同步。

在保证本地multiframe时钟同步之后,之后对于frame同步的处理都是相同的。

\subsection{Subclass 1类设备的本地multiframe时钟同步}

Subclass 1类设备的本地multiframe时钟同步主要是通过对齐外部的SYSREF信号。
需要处理的是两个问题,一个是SYSREF信号到达各个设备的同步问题,一个是设备自己的本地multiframe时钟与SYSREF信号对齐的时机问题。

\subsubsection{SYSREF信号同步}

针对第一个问题,协议要求所有设备的本地multiframe时钟必须要相同,但是SYSREF也是一个时钟信号,在传输和产生过程中,也会产生一定的时延,比如布线布局上存在的时延差问题。
所以在Subclass 1类设备中通过自己的设备时钟,当SYSREF信号到达后,产生指定数量的功能性时延后再对本地multiframe时钟进行对齐,以保证各个设备的本地multiframe时钟是同步的。
而对于时延数量的设置需要根据具体的电路设计来确定。

\subsubsection{本地multiframe时钟与SYSREF对齐时机}

针对第二个问题,协议并没有强制设定具体的对齐时机,但给出了一些建议。
发送端和接收端设备需要有能力决定是否校正本地frame和multiframe时钟到SYSREF脉冲。
这一功能实现的细节由设计者决定,但是有以下三种可能的选项供选择。

\begin{itemize}
\item 每一个SYSREF脉冲需要被设备验证,以决定现有的本地multiframe时钟相位和frame时钟是否需要对齐。
\item 一个设备可以通过设备的输入引脚或者控制接口命令来决定是否用下一个收到的SYSREF脉冲强制对齐本地multiframe时钟和本地frame相位。
\item 一个设备可以通过设备的输入引脚或者控制接口命令来决定是否忽略所有的SYSREF脉冲。
\end{itemize}

\subsection{Subclass 2设备的本地multiframe时钟同步}

Subclass 2类设备不同于Subclass 1类设备,Subclass 2类设备的本地multiframe时钟对齐要更加的复杂,主要依靠的是接收端和发送端反复的协商完成同步。
Subclass 2类设备由于没有共同的外部时钟源,所以本地multiframe时钟的同步完全依赖于SYNC~接口的信号变化,并且对于发送端和接收端的处理存在主从的区别。
需要处理的也是两个问题,一是最初链接建立的本地multiframe时钟同步问题,二是重新要求对齐的时机问题。

\subsubsection{建立本地multiframe时钟同步方法}

针对第一个问题,需要明确设备的主从关系,由主设备发起对本地multiframe时钟的调整请求。
发送端设备作为主设备,接收端设备作为从设备,目标就是通过发送端设备收到的SYNC~变化边界通过对比发送端设备本地的本地multiframe时钟边界判断是否需要调整。
由于之前协议的规定,接收端改变SYNC~信号的时机是来自接收端的本地multiframe时钟,这就意味着发送端得到SYNC~变化实质上是接收端的本地multiframe时钟边界。
当发送端收到这个信号后就会将这个信号对自己的本地multiframe时钟进行比较,从而判断是否符合同步条件,是否需要调整。
当发送端设备准确判断出需要调整时,会在接下来传输的ILA序列中添加调整参数,这几个参数是ADJCNT(调整步数)、ADJDIR(调整方向)、PHADJ(调整使能)。
这种情况下发送端不会发送具体数据,纯粹是利用ILA序列进行本地multiframe时钟的同步对齐。
而接收端设备在收到ILA序列中对于调整的需求时就将对自己的本地multiframe时钟进行调整,完成调整后在新的本地multiframe时钟上升沿在SYNC~接口发送一个错误报告。
发送端口又会根据这个错误报告的本地multiframe时钟边界判断是否需要继续调整,是否在合适的范围呢。
如此循环直到同步建立,当T发送端认为同步成功后就会发送具体的数据,并置低PHADJ。

这样的调整机制是建立在两个基础上。

\begin{itemize}
\item 一是发送端能够精确的比较SYNC~变化和本地multiframe时钟的差异,于是需要在发送端建立了检测分辨率和检测间隙机制。实际上就是对本地设备时钟的精度有一定的要求,能够准确判断本地multiframe时钟和SYNC~信号变化之间的相位差的差距具体时延数。而本地multiframe时钟对齐的精确度也是由这一机制来保证,设备时钟的精确度直接决定了检测的精度。
\item 二是接收端能够根据收到的调整数据对本地相位进行调整,于是在接收端建立了调整分辨率和调整时钟机制,调整时钟直接决定了调整的范围和精度。
\end{itemize}

事实上,光有机制的保证还是不够的,还需要调整算法的支持,以及调整范围、极限的设置。
时钟一次性调整的范围是有极限的,最大不会超过16个调整时钟的周期。
并且调整的方向也是会变化的,如何避免在调整过程中差异发散而非收敛,如何能够使调整尽快收敛进行传输也是需要设计者考虑的问题。

\subsubsection{重新对齐时机问题}

经过一定时间的正常传输后难免产生时钟的偏移问题,累计误差从而导致本地multiframe时钟不同步,这时就需要考虑重新对齐的问题。
发送端的SYNC~相位监视窗口会始终去判断是否需要重新对齐,一旦相位变化超出预设的窗口就表明延时丢失,需要重新发起初始化,来对齐本地multiframe时钟。
重新对齐的时机取决于使用者的配置,发起对齐的方法就是重新进行初始化循环,通过迭代的方法完成本地multiframe时钟对齐。

\bibliography{../../bib/serdes}
\end{document}
