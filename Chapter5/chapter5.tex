%%This is a very basic article template.
%%There is just one section and two subsections.
\documentclass[UTF8]{ctexart}
\title{第五章 JESD204B接收端传输层设计}
\author{陈登}
\date{\today}

\bibliographystyle{plain}
\usepackage{graphicx}
\usepackage{float}
\usepackage{amsmath}
\usepackage{geometry}
\usepackage{fontspec}
\usepackage{algorithm}
\usepackage{algorithmicx}
\usepackage{algpseudocode}

\geometry{a4paper,centering,scale=0.9}
\usepackage[format=hang,font=small,textfont=it]{caption}
\usepackage[toc,page,title,titletoc,header]{appendix}
\usepackage[nottoc]{tocbibind}

\begin{document}

\section{JESD204B接收端传输层设计}

传输层的主要是作为一个承上启下的层面,将下层经过解码、解扰、对齐等操作的序列恢复成发端赋予的具体的转换器数据。
在数据链路层完成了对信道信息的提取后,剩余的就是具体的样本数据,主要包含的是通道位置、样本数等信息。

而对具体数据包的拆包主要是依据各个lane的配置信息,配置信息包含了:
\begin{itemize}
  \item 单个link单个帧时钟周期包含的控制字节数,用CF表示。
  \item 单个转换器样本包含的控制比特数,用CS表示。
  \item 单个lane所包含的转换器数,用L表示。
  \item 单个设备包含的转换器数量,用M表示。
  \item 转换器的分辨率,用N表示。
  \item 单个帧周期单个转换器传输的样本数,用S表示。
\end{itemize}

传输层的具体工作就是根据协议的约定,通过已知的配置信息,将数据链路层得到的数据解包对应到正确的转换器和控制器上。

\subsection{}


\subsection{}

\subsection{}

\subsection{}

\bibliography{../../bib/serdes}
\end{document}
