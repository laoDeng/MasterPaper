\documentclass{beamer}
\usepackage[UTF8,noindent]{ctexcap}
\usepackage{beamerthemesplit}
\usepackage{amsmath}
\usetheme{Berlin}
\newtheorem{formula}{公式}

\title{基于JESD204B的Serdes接口中接收电路 \\ 设计研究}
\subtitle{论文答辩}
\author{陈登 \\ 导师:姚亚峰}
\date{\today}
\begin{document}

\begin{frame}
	\titlepage
\end{frame}

\begin{frame}{目录}
	\tableofcontents
\end{frame}

\section{研究背景}

\begin{frame}{研究背景}{JESD204B接口介绍}
在通信系统中,尤其是无线通信系统,高速AD转换芯片的地位非常重要。伴随着通信系统的传输速率不断飞速增长,传统的AD数据接口,如USB、SPI、I2C,已经远远无法满足在更高速条件下信号传输的需求。

于是一种新的接口技术,JESD204B应运而生,逐渐成为高速AD芯片上的必备接口,在实际中有着广泛的应用。
\end{frame}

\begin{frame}{研究背景}{JESD204B协议主要解决的问题}
  \begin{itemize}
    \item 传输高频无线数字信号需要很高的高速传输
    \item 所传输的数据需要适用于ADC、DAC的工作方式
    \item 各大厂商标准化的支持
  \end{itemize}
\end{frame}

\begin{frame}{研究背景}{JESD204B协议的特点}
  \begin{description}
    \item [新颖] 协议最早制定在2012年,属于硬件接口中的新成员,采用了串行设计。
    \item [高速] 协议规定在子类1条件下能够达到单通道12.5Gbps的传输速率。
    \item [专业] 协议是专门针对ADC、DAC芯片传输需求设计的,充分考虑信号的各种同步、传输情况。
    \item [通用] 协议已经实现在各大芯片公司的高端芯片中,如AD、TI等。
  \end{description}
\end{frame}

\begin{frame}{研究背景}{存在的问题}
现在国内市场上还很少能够看到自主生产的、拥有知识产权的、带有JESD204B接口的ADC、DAC芯片。并且很多现有的芯片并没有采用最新的JESD204B协议。

本课题研究的JESD204B接口接收端电路能够实际应用成为完整JESD204B接口的一部分,具有一定的价值。
\end{frame}

\section{JESD204B协议接收部分}

\begin{frame}{JESD204B协议接收部分}{接收端系统框图}
	\begin{figure}
	\centering
	\includegraphics[scale=1]{./img/recv_layer.pdf}
	\caption{接收端系统框图}
	\end{figure}
  \begin{description}
  \item [物理层] SerDes接收端,采用均衡、CDR、CML技术保证高速串行接收。
  \item [链路层] 8B/10B解码器、解扰器、控制字符。
  \item [传输层] 解帧器。
  \end{description}
\end{frame}

\begin{frame}{研究内容和目标}{研究内容}
	\begin{description}
	\item [接收端链路层设计] 8B/10B解码器、解扰器、同步检测。
	\item [接收端传输层设计] 解帧器。
	\end{description}
\end{frame}


\begin{frame}{研究内容和目标}{研究目标}
	\begin{itemize}
	\item 设计出符合JESD204B协议标准的SERDES接收端的数据链路层和传输层电路设计,完成RTL仿真及综合。
	\item 在SMIC 180nm工艺标准下,固定导线负载标准下。
		\begin{itemize}
		\item 整体工作频率达到500Mhz。
		\item 单元面积小于10000$\mu m^2$。
		\end{itemize}
	\item 能够通过配置寄存器调整传输模式、传输速度等,能够通过接口报告错误,具有一定的差错处理能力。
	\end{itemize}
\end{frame}

\begin{frame}{研究内容和目标}{拟解决关键问题}
为完成一个基于JESD204B协议的SERDES接收端电路设计,需要解决的关键技术有
	\begin{itemize}
	\item 数据链路层电路设计。根据设计需要,设计所需的各个子模块,如8B/10B解码器,解扰器,同步检测模块。
	\item 传输层电路设计。通过链路层获得的数据进一步将整合在一个包中的数据解开,对其中发生的错误进行纠正或报错。
	\item 通过最终的连接仿真,要达到正确传输的一般功能外,还要达到预期的技术指标。
	\end{itemize}
\end{frame}

\section{接收部分具体设计}

\begin{frame}{选题研究方案、可行性分析及特色}{研究方法}
通过对已有研究成果和JESD204B协议的理解,分析以及对现存的各种SERDES接口电路设计的探索,力求使系统工作频率高,芯片面积小,功能达到预期要求,从数据链路层和传输层两个层面对课题进行设计研究。
\end{frame}

\begin{frame}{选题研究方案、可行性分析及特色}{技术路线}
基于JESD204B的Serdes接口中接收电路设计主要由输入信号的有数据链路层和传输层两个部分及测试部分构成。
	\begin{itemize}
	\item 数据链路层主要负责解码、解扰、同步等工作,重点在于对每个字节的处理。
	\item 传输层主要负责解帧的工作,重点在于对于一串字节的处理。
	\item 外围测试电路设计,主要为伪随机序列生成、输入信号生成,来模拟发送端的数据完成逻辑测试。
	\end{itemize}
\end{frame}

\begin{frame}{选题研究方案、可行性分析及特色}{实验方案}
	\begin{itemize}
	\item 通过Verdi3作为前期的设计Debug工具。
	\item 设计通过Modelsim的RTL级仿真来确保逻辑的工作正确。
	\item 设计复杂的testbench来完成可靠性的测试,依据协议规定还需要使用伪随机序列来对正确性进行测试。
	\item 完成RTL级仿真后再使用Design Compiler进行门级综合,使用SMIC 180nm工艺,固定导线负载,通过综合结果判断设计性能。
	\end{itemize}
\end{frame}

\begin{frame}{选题研究方案、可行性分析及特色}{特色}
	\begin{itemize}
	\item 本课题主要关注业界最新的JESD204B协议,其中含有新的数据链路层控制逻辑,并且要保证向下兼容。
	\item 电路设计层级提升到综合级别,相较于RTL级别的设计能更直观的判断电路的好坏,并且采用的是ASIC的设计思路,直接使用的是厂方工艺库,更贴合业界需求。
	\item 在保证面积和功耗的前提下,在特定的工艺级别下,能够达到较高的工作频率。
	\end{itemize}
\end{frame}

\section{总结}

\begin{frame}{学位论文工作计划}{计划}
	\begin{description}
	\item[2014.11-2014.12] 协议理解、文献阅读
	\item[2014.12-2015.01] 数据链路层设计
	\item[2015.01-2015.02] 传输层设计
	\item[2015.02-2015.03] 数据链路层、传输层联调
	\item[2015.03-2015.05] 撰写论文
	\end{description}
\end{frame}

\end{document}
