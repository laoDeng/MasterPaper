%%This is a very basic article template.
%%There is just one section and two subsections.
\documentclass[UTF8]{ctexart}
\title{论文计划}
\author{陈登}
\date{\today}

\bibliographystyle{plain}
\usepackage{graphicx}
\usepackage{float}
\usepackage{amsmath}
\usepackage{geometry}
\usepackage{fontspec}
\usepackage{algorithm}
\usepackage{algorithmicx}
\usepackage{algpseudocode}

\newtheorem{aim}{目标}
\newtheorem{deadline}{时间}

\geometry{a4paper,centering,scale=0.9}
\usepackage[format=hang,font=small,textfont=it]{caption}
\usepackage[toc,page,title,titletoc,header]{appendix} 
\usepackage[nottoc]{tocbibind}

\begin{document}

\section{序言}

\begin{aim}{}完成一些基础性的概述内容,需要阅读一些现有硕博士论文的描述,结合本课题进行描述。\end{aim}
\begin{deadline}{}2014.12.15-2014.12.17\end{deadline}

\subsection{课题来源}
参考开题报告和相关论文,结合项目内容,阐述开题来源。
\subsection{国内外发展现状及优缺点}
参考开题报告和相关论文,针对SerDes接口在国内缺乏研发进行阐述。
\subsection{研究特点}
重点描述JESD204B协议的新颖,国内缺乏研究基础。
\subsection{论文结构}
根据提纲描述具体论文机构。

\section{基本原理及工具}

\begin{aim}{}就设计中可能需要设计的原理和工具进行简介,突出芯片设计的流程。\end{aim}
\begin{deadline}{}2014.12.18-2014.12.22\end{deadline}

\subsection{SerDes接口基本原理}
参考协议,概括SerDes接口具体工作方式。
\subsection{芯片设计流程介绍}
简介ASIC设计的主要流程及重点难点。
\subsection{设计工具介绍}
对设计中使用到的DC、ModelSim、Verdi进行简单介绍。

\section{SerDes接收端整体结构介绍}

\begin{aim}{}根据协议规定内容,对接收端的各个部分做概述,并绘制系统框图,确定最终设计目标。\end{aim}
\begin{deadline}{}2014.12.23-2015.12.26\end{deadline}

\subsection{总体框架}
接收端的整体的从串行数据开始到传输结果输出的流程描述,即结构框图。
\subsection{数据链路层}
数据链路层负责的字节处理流程的描述,从输入到输出需要进过的模块及结果。
\subsection{传输层}
传输层负责的字节流处理流程的描述,从链路层传送出来的数据进一步处理的方法。
\subsection{设计指标}
即对设计芯片面积、功耗、仿真结果、工作频率的描述,在特定工艺下的效果估计。

\section{SerDes接收端数据链路层设计}

\begin{aim}{}根据协议规定内容,对数据链路层的各个子模块进行设计,分模块阐述设计思路、方法、关键代码。\end{aim}
\begin{deadline}{}2014.12.26-2015.1.10\end{deadline}

\subsection{8B/10B解码器设计}
数据链路层核心模块,结合之前设计思路,详细描述该设计,另外对重新对代码仿真已验证。

\subsection{解扰器设计}
阅读协议和已有研究结果,对解扰器的设计思路进行描述,另需写代码完成仿真验证。

\subsection{Frame/Lane对齐字符检测设计}
阅读协议,绘制出对齐检测的状态转换图,写代码完成仿真验证。

\section{SerDes接收端传输层设计}

\begin{aim}{}根据协议规定内容,对传输层的各个子模块进行设计,分模块阐述设计思路、方法、关键代码。\end{aim}
\begin{deadline}{}2015.1.20-2015.2.2\end{deadline}

\subsection{解帧器设计}
解帧器是接收端传输层的主要功能模块,重点还是在于状态机的设计描述,另需写代码完成仿真验证。

\subsection{差错检验设计}
差错检验负责全局的差错处理,以及向应用层汇报下层运行状态。

\section{设计结果验证}

\begin{aim}{}对各个子模块的设计结果进行详细说明,包括大小、功耗、性能。级联仿真后的运行效果进行评估。仿真结果的描述可以在之前模块设计同时进行。\end{aim}
\begin{deadline}{}2015.2.3-2015.2.10\end{deadline}

\subsection{数据链路层仿真结果}
根据之前的链路层模块综合仿真得到的结果进行描述和分析。
\subsubsection{8B/10B解码器仿真结果}
\subsubsection{解扰器仿真结果}
\subsubsection{Frame/Lane对齐字符检测仿真结果}

\subsection{传输层仿真结果}
根据之前的传输层模块综合仿真得到的结果进行描述和分析。
\subsubsection{解帧器仿真结果}
\subsubsection{差错检验仿真结果}

\subsection{综合仿真结果}
将之前的模块级联后得到整个接收端的仿真结果。

\section{结论}

\begin{aim}{}将主要内容再进行复述,特别要注明论文的创新点。同时,指出研究工作还需要改进的地方,或者今后继续努力的方向。\end{aim}
\begin{deadline}{}2014.2.11-2015.2.15\end{deadline}

\section{摘要、格式、参考文献及其他}

\begin{aim}{}根据主体完成的内容撰写摘要等内容,并提交草稿。\end{aim}
\begin{deadline}{}2014.2.16-2015.2.20\end{deadline}

\end{document}

