\documentclass[UTF8]{ctexart}
\title{中文摘要}
\author{陈登}
\date{\today}

\bibliographystyle{plain}
\usepackage{graphicx}
\usepackage{float}
\usepackage{amsmath}
\usepackage{geometry}
\usepackage{fontspec}
\usepackage{algorithm}
\usepackage{algorithmicx}
\usepackage{algpseudocode}

\geometry{a4paper,centering,scale=0.9}
\usepackage[format=hang,font=small,textfont=it]{caption}
\usepackage[toc,page,title,titletoc,header]{appendix}
\usepackage[nottoc]{tocbibind}

\begin{document}

JESD204B协议是高速模数转换芯片、高速模数转换芯片同高速信号处理器之间进行通信的协议之一,它采用了SerDes作为其物理层标准,通过差分串行设计保证了高速的数据传输。正由于其高速的特点,基于JESD204B协议的通信接口得到越来越多工程设计人员的青睐,并广泛应用于各类高速模数转换芯片、高速模数转换芯片以及FPGA、DSP等高速信号处理器上。

对于样本数据传输方面,相对于传统的通信接口,基于JESD204B协议的通信接口很好的解决了转换器分配、多芯片同步、多转换器同步、多链路传输等相关问题,由于采用了串行传输的设计,克服了并行接口布线复杂的问题,降低了系统的复杂度提高了稳定性。JESD204B协议公布的时间比较短,国内还缺乏基于该协议的通信接口设计实现及相关研究。本文根据ASIC设计思路,旨在提供一种基于JESD204B协议的通信接口接收端的设计思路和方法,包括了协议中的数据链路层。在保证基本符合协议标准的前提下,针对实际工程应用做出了一定改进。

本文的主要完成的工作是设计通信接口的接收端电路,包括8B/10B解码器设计、解扰器设计、码群同步状态机设计、初始化lane同步状态机设计、初始化帧同步状态机设计、数据流模块设计以及相关配套配置电路设计。通过分析协议接收端相关设计要求,结合已有相关芯片数据手册,得到合理的设计思路。具体实现采用Verilog语言进行RTL级设计,Verdi3进行错误检查,再通过Modelsim进行仿真,最后使用Design Compiler配合SMIC 180nm工艺库进行综合,最终得到逻辑仿真结果和电路综合结果,完成部分芯片前端设计工作。其中8B/10B解码器设计方法相较之前相关论文的设计方法,在面积和工作频率上有一定的优势。实现的码群同步状态机单元面积为1500$\mu m^2$,工作频率可达1GHz;初始化lane同步状态机单元面积为8700$\mu m^2$,工作频率可达162MHz;数据流模块单元面积为5000$\mu m^2$,工作频率可达500MHz;8B/10B解码器模块单元面积为1759$\mu m^2$,工作频率可达224MHz。

\bibliography{../../bib/serdes}
\end{document}
