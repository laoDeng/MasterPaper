\documentclass[UTF8]{ctexart}
\title{致谢}
\author{陈登}
\date{\today}

\bibliographystyle{plain}
\usepackage{graphicx}
\usepackage{float}
\usepackage{amsmath}
\usepackage{geometry}
\usepackage{fontspec}
\usepackage{algorithm}
\usepackage{algorithmicx}
\usepackage{algpseudocode}

\geometry{a4paper,centering,scale=0.9}
\usepackage[format=hang,font=small,textfont=it]{caption}
\usepackage[toc,page,title,titletoc,header]{appendix}
\usepackage[nottoc]{tocbibind}

\begin{document}

在两年的研究生生活即将结束之际,回想这段日子的点点滴滴,我的心中充满了自豪与感激。
求知的道路是艰难的,既有来自学术的压力,也有对于生活的无奈,在这段日子里,是导师的指导、家人的关爱、朋友的支持,帮助我客服了一切困难、勇往直前,在此谨向他们致以最真诚的感谢!

我要衷心的感谢我的硕士导师姚亚峰教授。
姚老师严谨的治学态度、诲人不倦的师德、一丝不苟的工作作风和所提倡的自由的学术氛围,深深的影响了我,使我终生受益。
在我参加项目研发过程中,姚老师经常与我进行交流,对我的工作和生活情况非常关心,对于我提出的问题给予了很好的回复,给我提供了很多帮助。
他不厌其烦地教导我如何用科学的方法做科研,并毫无保留地与我分享他丰富的生活和科研经验,这对于我来说是一笔非常宝贵的财富。
本论文也是在姚老师的悉心指导下完成的,他不仅为我们提供了良好的科研环境,并且在论文选题、研究内容及应用,再到论文撰写的细节刚面都给予了细致的指导和建设性的意见,使我能圆满而顺利的完成硕士研究生的科研与学习任务。
在本论文完成之际,谨向姚亚峰老师致以深深的敬意和由衷的感谢。

感谢在一起度过研究生生活的教二楼509室的各位同门,正是有了你们的帮助和支持,我才能解决一个又一个的难题,解答一个又一个的疑惑,直至本文顺利完成。
特别感谢霍兴华同学的热情帮助,他在我写论文的过程中,对于一些关键性的问题给予了我很多帮助和建议,他对芯片设计的热爱深深的影响了我,通过和他探讨问题使我受益良多。感谢我的室友龚文佳、杜永超同学,各自从遥远的城市来到这里,是你们的支持、鼓励和帮助让我能够自信的面对各种挑战,愿友谊长存!

感谢全体机电学院的老师和研究生两年来的各位任课老师。感谢他们对我学习和科研中无私的奉献、关怀和帮助。
感谢我的母校————中国地质大学(武汉),给我提供了良好的学习环境和成长条件,以及六年来对我的教育和培养,使我从一个懵懂的高中毕业生成长为了一个能够迈向广阔社会舞台的合格人才。

我要特别感谢我的家人,感谢他们对我的支持、理解和鼓励,他们使我永远的动力,再次感谢他们。

最后衷心的感谢对本论文进行评阅的老师和答辩会的评委老师们,你们辛苦了,感谢你们。

再一次感谢一切关心和帮助过我的人,祝他们身体健康、万事如意。

\bibliography{../../bib/serdes}
\end{document}
