\documentclass[UTF8]{ctexart}
\title{作者介绍}
\author{陈登}
\date{\today}

\bibliographystyle{plain}
\usepackage{graphicx}
\usepackage{float}
\usepackage{amsmath}
\usepackage{geometry}
\usepackage{fontspec}
\usepackage{algorithm}
\usepackage{algorithmicx}
\usepackage{algpseudocode}

\geometry{a4paper,centering,scale=0.9}
\usepackage[format=hang,font=small,textfont=it]{caption}
\usepackage[toc,page,title,titletoc,header]{appendix}
\usepackage[nottoc]{tocbibind}

\begin{document}

陈登,男,浙江宁波人。2013年7月毕业于中国地质大学(武汉)机械与电子信息学院,获通信工程学士学位,同年面试推荐为中国地质大学(武汉)电子与通信工程方向研究生。
研究方向为大型数字电路设计。

研究生期间共学习了11门课程,总学分为28分,各科平均成绩为84分。
具体课程:知识产权法、现代数字图像处理与分析、信息检索与利用、网络安全技术、数字系统设计、现代数字信号处理、工程硕士英语、自然辩证法概论、随机过程、数字通信、VisualC++程序设计,无不及格科目。

研究生两年期间,在导师的指导下,顺利完成了学习及相关科研任务,参与了基于JESD204B协议的通信接口接收端电路设计。
对自己的专业知识和动手能力提高有很大的帮助。

硕士期间发表学术论文1篇:
JESD204B接口协议中的8B/10B解码器设计,2014年10月,电视技术。

\bibliography{../../bib/serdes}
\end{document}
